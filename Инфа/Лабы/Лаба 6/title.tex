\documentclass[12pt, a4pape]{article}

\usepackage[utf8]{inputenc}
\usepackage{fancyhdr}
\usepackage{caption}
\usepackage{multicol}
\usepackage[english, russian]{babel}
\usepackage{amsmath}
\usepackage{amssymb}
\usepackage{makecell} % Для выравнивания внутри ячйки
\usepackage{graphicx} % Вставка картинок правильная
\usepackage{float} % "Плавающие" картинки
\usepackage{wrapfig} % Обтекание фигур (таблиц, картинок и прочего)
\usepackage{array}
\usepackage{booktabs}

\usepackage{geometry}
\geometry{  
    a4paper,
    top=8mm, 
    right=16mm, 
    bottom=20mm, 
    left=8mm
}

\pagestyle{fancy}
\fancyhf{} % Очистка всех полей заголовков
\renewcommand{\headrulewidth}{0} % Убираем линию вверху страниц

\setlength{\columnsep}{29pt}

\begin{document}
    \begin{center}
    \Large ФЕДЕРАЛЬНОЕ ГОСУДАРСТВЕННОЕ АВТОНОМНОЕ ОБРАЗОВАТЕЛЬНОЕ УЧРЕЖДЕНИЕ ВЫСШЕГО ОБРАЗОВАНИЯ 
«НАЦИОНАЛЬНЫЙ ИССЛЕДОВАТЕЛЬСКИЙ УНИВЕРСИТЕТ ИТМО»\\
    
    Факультет программной инженерии и компьютерной техники\\
    \hfill  
    
    \vspace{7cm}
    \Large Лабораторная работа №6 \\
    Работа с системой компьютерной вёрстки \TeX\\
    Вариант: 9\\
    \end{center}
    
    \vspace{6.5cm}
     
    \begin{flushright}
    \textit{Выполнил:}\\
    Караганов Павел Эдуардович\\
    Группа P3110\
    
    \textit{Проверил:}\\
    Балакшин П. В. (ординарный доцент)\\
    Рыбаков С. Д. (преподаватель  практик)\\
    \end{flushright}
     
    \vfill
    
    \begin{center} Санкт-Петербург, 2024 \end{center}
    \newpage
\end{document}